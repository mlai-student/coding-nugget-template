%%%%%%%%% DO NOT ALTER THIS, OPTIONAL PART STARTS BELOW %%%%%%%%%%%%

%%% Disable some standard ACM things that we don't want to see
\AtBeginDocument{%
  \providecommand\BibTeX{{%
    \normalfont B\kern-0.5em{\scshape i\kern-0.25em b}\kern-0.8em\TeX}}}

\setcopyright{none}
\copyrightyear{none}
\acmYear{none}
\acmDOI{none}


%%% ML2R colors
\definecolor{ml2rblu}{rgb}{0.02,0.27,0.45}
\definecolor{ml2ryel}{rgb}{0.98,0.72,0.18}
\definecolor{ml2rgrn}{rgb}{0.50,0.71,0.18}
\definecolor{ml2rtrq}{rgb}{0.00,0.57,0.57}

%%% not yet imported by ACM style
% \usepackage{amssymb}
\usepackage{amsthm}
\usepackage{mathtools}
\usepackage{bm}
\usepackage[font=footnotesize]{subfig}
\usepackage{booktabs}
\usepackage{array}
\usepackage{fixltx2e}
\usepackage{nicefrac}

\usepackage{wasysym}

\usepackage{tikz}
\usetikzlibrary{shapes.geometric, positioning}

\usepackage{algorithm}
\usepackage[noend]{algpseudocode}
\renewcommand{\algorithmiccomment}[1]{{\color{DarkRed!90}// #1}}

\usepackage{hyperref}
\hypersetup{%
  colorlinks=true,
  urlcolor=ml2rtrq,
  linkcolor=ml2rtrq,
  citecolor=ml2rtrq,
  bookmarks=false}


\usepackage{fancyvrb} 
\usepackage{listings}
% Python style / environment for highlighting in "figures"
\lstdefinestyle{pythonstyle}{%
  language=Python,
  tabsize=4,
  backgroundcolor=\color{Gray!10},
  basicstyle=\ttfamily\scriptsize,
  stringstyle=\color{ForestGreen},
  keywordstyle=\color{BlueViolet},
  commentstyle=\itshape\color{DarkRed!90},
  identifierstyle=,
  emphstyle=\color{Blue},
  frame=lines,	
  showstringspaces=false,
  morekeywords={range, len, self, other, lambda, from, import, as, False, True, 
  enumerate, xrange, map, list, set, float, int, min, max, sorted, None},
  fancyvrb=true,
}
\lstnewenvironment{python}[1][]{\lstset{style=pythonstyle,#1}}{}

% Python style / environment for highlighting in body
\lstdefinestyle{pythonstyletxt}{%
  language=Python,
  tabsize=4,
  basicstyle=\ttfamily\small,
  stringstyle=\color{ForestGreen},
  keywordstyle=\color{BlueViolet},
  commentstyle=\itshape\color{DarkRed!90},
  identifierstyle=,
  emphstyle=\color{Blue},
  %frame=l,	
  xleftmargin=1em,
  showstringspaces=false,
  morekeywords={range, len, self, other, lambda, from, import, as, False, True, 
  enumerate, xrange, map, list, set, float, int, min, max, sorted, with, None},
}
\lstnewenvironment{pythontxt}[1][]{\lstset{style=pythonstyletxt,#1}}{}

% Python style / environment for small size highlighting in body
\lstdefinestyle{pythonstyletxtsmall}{%
  language=Python,
  tabsize=4,
  basicstyle=\ttfamily\scriptsize,
  stringstyle=\color{ForestGreen},
  keywordstyle=\color{BlueViolet},
  commentstyle=\itshape\color{DarkRed!90},
  identifierstyle=,
  emphstyle=\color{Blue},
  %frame=l,	
  xleftmargin=1em,
  showstringspaces=false,
  morekeywords={range, len, self, other, lambda, from, import, as, False, True, 
  enumerate, xrange, map, list, set, float, int, min, max, sorted, None},
}
\lstnewenvironment{pythontxtsmall}[1][]{\lstset{style=pythonstyletxtsmall,#1}}{}


%%%%%%%% OPTIONAL %%%%%%%%%%%%%%%

% only used for fig colors, may be removed if you don't need it
\usepackage{tikz}
\usetikzlibrary{shapes.geometric, positioning}

%%% define convience commands
% python stuff
\newcommand{\Py}{\emph{Python}}
\newcommand{\PY}{\emph{Python }}
\newcommand{\NP}{\emph{NumPy }}
\newcommand{\Np}{\emph{NumPy}}
\newcommand{\SP}{\emph{SciPy }}
\newcommand{\Sp}{\emph{SciPy}}
\newcommand{\MP}{\emph{Matplotlib }}
\newcommand{\Mp}{\emph{Matplotlib}}
\newcommand{\NX}{\emph{NetworkX }}
\newcommand{\Nx}{\emph{NetworkX}}
\newcommand{\QT}{\emph{QuTiP }}
\newcommand{\Qt}{\emph{QuTiP}}

% text highlighting
\newcommand{\alert}[1]{\emph{#1}}
\newcommand{\ALERT}[1]{{\color{red}#1}}
\newcommand{\keyword}[1]{\emph{\texttt{\color{blue}#1}}}

% vectors, matrices, and tensors
\renewcommand{\vec}[1]{\bm{#1}}
\newcommand{\mat}[1]{\bm{#1}}
\newcommand{\ten}[1]{\bm{\mathcal{#1}}}

% transposition
\newcommand{\trn}[1]{#1^\intercal}

%%% inner, outer products and squared Euclidean norms/distances
\newcommand{\IPA}[2]{\Bigl \langle #1 \bigm| #2 \Bigr \rangle}
\newcommand{\Ipa}[2]{\bigl \langle #1 \bigm| #2 \bigr \rangle}
\newcommand{\ipa}[2]{\left \langle #1 \,\middle|\, #2 \right \rangle}
%\newcommand{\ipt}[2]{#1^{\scriptscriptstyle T} #2}
\newcommand{\ipt}[2]{\trn{#1} #2}
%\newcommand{\ipt}[2]{#1^\intercal #2}
%\newcommand{\ipt}[2]{#1^\top #2}
%\newcommand{\ipt}[2]{#1^\mathsf{T} #2}
\newcommand{\opt}[2]{#1 \trn{#2}}
\newcommand{\nrm}[1]{\bigl \lVert #1 \bigr \rVert^2}
\newcommand{\NRM}[1]{\Bigl \lVert #1 \Bigr \rVert^2}
\newcommand{\dsq}[2]{\bigl \lVert #1 - #2 \bigr \rVert^2}
\newcommand{\DSQ}[2]{\Bigl \lVert #1 - #2 \Bigr \rVert^2}
\newcommand{\dist}[2]{\bigl \lVert #1 - #2 \bigr \rVert}
\DeclareMathOperator*{\vbar}{\Bigr\rvert}

%%% inner and outer product for indexed vectors
\newcommand{\iipt}[2]{\trn{#1} #2^{\phantom{\intercal}}}
\newcommand{\oipt}[2]{#1^{\phantom{\intercal}} \trn{#2}}

%%% trace and rank operator
\newcommand{\tr}[1]{\operatorname{tr} \bigl [ #1 \bigr ]}
\newcommand{\TR}[1]{\operatorname{tr} \Bigl [ #1 \Bigr ]}
\newcommand{\rk}[1]{\operatorname{rk} \bigl [ #1 \bigr ]}
\newcommand{\RK}[1]{\operatorname{rk} \Bigl [ #1 \Bigr ]}
\newcommand{\diag}[1]{\operatorname{diag} \bigl [ #1 \bigr ]}
\newcommand{\DIAG}[1]{\operatorname{diag} \Bigl [ #1 \Bigr ]}

%%% sets and optimization routines
\newcommand{\set}[1]{\mathcal{#1}}
\newcommand{\st}{\operatorname{s.\!t.}}
\newcommand{\amin}[1]{\operatorname*{argmin}_{#1}}
\newcommand{\amax}[1]{\operatorname*{argmax}_{#1}}

\newcommand{\submax}[1]{#1_{\max}}

%%% conditional independence and probability
\newcommand{\ci}{\perp\!\!\!\perp}
\newcommand{\prob}[1]{p\bigl( #1 \bigr)}
\newcommand{\cprob}[2]{p\bigl( #1 \mid #2 \bigr)}

%%% bras and kets
\newcommand{\ket}[1]{\vert {#1} \rangle}
\newcommand{\Ket}[1]{\big\vert {#1} \big\rangle}
\newcommand{\bra}[1]{\langle {#1} \vert}
\newcommand{\Bra}[1]{\big\langle {#1} \big\vert}
\newcommand{\braket}[2]{\langle {#1} \vert {#2} \rangle}
\newcommand{\Braket}[2]{\big\langle {#1} \big\vert {#2} \big\rangle}
\newcommand{\ketbra}[2]{\vert {#1} \rangle \langle {#2} \vert}
\newcommand{\Ketbra}[2]{\big\vert {#1} \big\rangle \big\langle {#2} \big\vert}

\newcommand{\hdash}{\operatorname{\,{}---{}\,}}
\renewcommand{\vdash}{\arrowvert}

